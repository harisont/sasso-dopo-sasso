% Options for packages loaded elsewhere
\PassOptionsToPackage{unicode}{hyperref}
\PassOptionsToPackage{hyphens}{url}
%
\documentclass[
]{article}
\usepackage{amsmath,amssymb}
\usepackage{lmodern}
\usepackage{iftex}
\ifPDFTeX
  \usepackage[T1]{fontenc}
  \usepackage[utf8]{inputenc}
  \usepackage{textcomp} % provide euro and other symbols
\else % if luatex or xetex
  \usepackage{unicode-math}
  \defaultfontfeatures{Scale=MatchLowercase}
  \defaultfontfeatures[\rmfamily]{Ligatures=TeX,Scale=1}
\fi
% Use upquote if available, for straight quotes in verbatim environments
\IfFileExists{upquote.sty}{\usepackage{upquote}}{}
\IfFileExists{microtype.sty}{% use microtype if available
  \usepackage[]{microtype}
  \UseMicrotypeSet[protrusion]{basicmath} % disable protrusion for tt fonts
}{}
\makeatletter
\@ifundefined{KOMAClassName}{% if non-KOMA class
  \IfFileExists{parskip.sty}{%
    \usepackage{parskip}
  }{% else
    \setlength{\parindent}{0pt}
    \setlength{\parskip}{6pt plus 2pt minus 1pt}}
}{% if KOMA class
  \KOMAoptions{parskip=half}}
\makeatother
\usepackage{xcolor}
\usepackage{graphicx}
\makeatletter
\def\maxwidth{\ifdim\Gin@nat@width>\linewidth\linewidth\else\Gin@nat@width\fi}
\def\maxheight{\ifdim\Gin@nat@height>\textheight\textheight\else\Gin@nat@height\fi}
\makeatother
% Scale images if necessary, so that they will not overflow the page
% margins by default, and it is still possible to overwrite the defaults
% using explicit options in \includegraphics[width, height, ...]{}
\setkeys{Gin}{width=\maxwidth,height=\maxheight,keepaspectratio}
% Set default figure placement to htbp
\makeatletter
\def\fps@figure{htbp}
\makeatother
\setlength{\emergencystretch}{3em} % prevent overfull lines
\providecommand{\tightlist}{%
  \setlength{\itemsep}{0pt}\setlength{\parskip}{0pt}}
\setcounter{secnumdepth}{-\maxdimen} % remove section numbering
\ifLuaTeX
  \usepackage{selnolig}  % disable illegal ligatures
\fi
\IfFileExists{bookmark.sty}{\usepackage{bookmark}}{\usepackage{hyperref}}
\IfFileExists{xurl.sty}{\usepackage{xurl}}{} % add URL line breaks if available
\urlstyle{same} % disable monospaced font for URLs
\hypersetup{
  pdftitle={Sasso dopo sasso - la mostra},
  hidelinks,
  pdfcreator={LaTeX via pandoc}}

\usepackage[top=1in, bottom=1in, left=1in, right=1in]{geometry}

\begin{document}

\Large
Nella sezione ``\textbf{Un mare di sassi}'' confluiscono reperti che
ricordano, per forma e colorazione, la specie ittiche del Mediterraneo,
dai grandi cetacei oramai estinti alla piú contemporanea sardina; una
carrellata che si conclude ipotizzando uno scenario futuribile
caratterizzato dalla contaminazione pesce-plastica.

Testimonianza ulteriore del forte interesse faunistico della curatrice
Katia Grillo é la serie ``\textbf{Uccellacci ed uccellini}'', la scelta
dei cui pezzi si ispira al neorealismo pasoliniano. Sezione piú ricca e
nutrita dell'intera esposizione, ospita diversi esemplari litici di
volatili domestici e selvatici. Al suo interno, per la prima volta parte
di un'esposizione aperta al pubblico, alcuni pezzi della collezione
privata fanese ``\textbf{Le galline pensierose}'', omaggio all'omonimo
testo del Malerba.

``\textbf{Sassi consunti}'' ospita invece ciottoli fluitati
dall'apparente ma convincente aspetto di manufatti, spostando
l'attenzione sul rapporto uomo-natura. Di particolare interesse il
fenomeno natura-cultura-natura, per cui il materiale grezzo, dopo la
lavorazione, é nuovamente smussato dalla periodica azione meccanica del
moto ondoso e levigato dall'azione congiunta del vento e della sabbia,
riassumendo dunque la \emph{facies} di elemento naturale.

Adatte ad un pubblico di ogni etá le due sezioni ``\textbf{Sassi
bestiali}'' e ``\textbf{Sassi espressivi}''. La prima, una collezione di
pietre zoomorfe, evidenzia la straordinaria biodiversitá del Comune di
Realmonte. La seconda, composta da ciottoli antropomorfi, rimanda,
tramite le innumerevoli espressioni facciali rappresentatevi, alle
diverse sfumature dell'emotivitá umana.

Lungi dal concentrarsi esclusivamente sull'estetica del ciottolo, la
mostra dá spazio anche agli aspetti pragmatici della pietra, presentando
un campionario piccolo ma significativo delle forme naturali che senza
alcun dubbio hanno ispirato gli esordi del design scandinavo e che
continuano a dettare i canoni stilistici del colosso svedese IKEA. La
sezione, intolata ``\textbf{Sassi utili}'', presenta sassi con una
funzione specifica, estrapolandoli saggiamente dal loro contesto d'uso
quotidiano.

La serie ``\textbf{Sassi vari}'', culminante nel gruppo ``\textbf{I
grandi sassi}'', presenta pietre aventi ciascuna una forte
individualitá, nella cui interpretazione il visitatore é volutamente
lasciato libero, assolvendo cosí ad una funzione proiettivo-catartica.

L'importante sezione ``\textbf{Altri sassi}'' intende invece indurre il
pubblico ad un atteggiamento di empatico rispetto verso ció che ad uno
sguardo frettoloso puó apparire materia inerte e inamovibile. Al
contrario, il gruppo ``\textbf{Sassi in piedi, sassi seduti}''
restituisce allo spettatore un'impressione di vivo dinamismo.

L'itinerario presenta inoltre una sezione interattiva, ``\textbf{Sassi di
scorta}'', in cui al vistatore viene offerta la possibilitá di
partecipare da protagonista alla catalogazione e all'inserimento in
mostra di ciottoli non ancora esposti.

\end{document}
